
% Magic Comments
% !TeX spellcheck = en_GB
% !TeX encoding = utf8
% !TeX program = pdflatex
% !TeX root = main.tex

% Package specific things

% TODO: Do we want to have equations just labeled with "(3.1)" or with 
% TODO: "equation 3.1".
%\crefname{equation}{equation}{equations} 
%\Crefname{equation}{Equation}{Equations}
\crefname{equation}{}{} 
\Crefname{equation}{}{}

\crefname{table}{table}{tables}
\Crefname{table}{Table}{Tables}

\crefname{figure}{figure}{figures}
\Crefname{figure}{Figure}{Figures}


% Definitions
\newcommand{\defn}{\ensuremath{:=}}
\newcommand{\revdefn}{=:}

% Sets
\newcommand{\R}{\mathbb{R}}
\newcommand{\F}{\mathbb{F}}
\newcommand{\N}{\mathbb{N}}
\newcommand{\Z}{\mathbb{Z}}
\newcommand{\C}{\mathbb{C}}
\newcommand{\Q}{\mathbb{Q}}
\newcommand{\T}{\mathbb{T}}

% Caligraphic
\newcommand{\cO}{\mathcal{O}}
\newcommand{\cZ}{\mathcal{Z}}
\newcommand{\cP}{\mathcal{P}}
\newcommand{\cI}{\mathcal{I}}
\newcommand{\cA}{\mathcal{A}}

% Function Spaces
\newcommand*{\pols}[2][]{\ensuremath{\cP^{#1}\left(#2\right)}}

% bold
\newcommand{\ZZ}{\ensuremath{\mathbf{Z}}}
\newcommand{\FF}{\ensuremath{\mathbf{F}}}
\newcommand{\XX}{\ensuremath{\mathbf{X}}}
\newcommand{\YY}{\ensuremath{\mathbf{Y}}}
\newcommand{\bfj}{\ensuremath{\mathbf{j}}}
\newcommand{\bfx}{\ensuremath{\mathbf{x}}}
\newcommand{\bfz}{\ensuremath{\mathbf{z}}}
\newcommand{\bfy}{\ensuremath{\mathbf{y}}}
\newcommand{\bfom}{\ensuremath{\mathbf{\omega}}}
\newcommand{\bfw}{\ensuremath{\mathbf{w}}}

% Set brackets
\newcommand{\set}[1]{\left\{#1\right\}}

% Make index set from 1 to (inclusive) m:
\newcommand*{\indexset}[2][1]{[#1:#2]}

% Make a colon: (Used for defining a function from A to B)
\newcommand*{\from}{\colon}

% Probability commands: Expectation, Variance, Covariance, Bias
\newcommand{\E}{\mathbb{E}}

% Vector Norm
\newcommand{\norm}[2][2]{\left\lVert#2\right\rVert_{#1}}

% Derivatives
\newcommand{\deriv}[2][x]{\frac{\mathrm{d}}{\mathrm{d}#1} (#2)}

% Partial derivatives
\newcommand{\pderiv}[2]{\frac{\partial #1}{\partial #2}}

% Integral 
\newcommand{\integral}[4]{\int\limits_{#1}^{#2}#3~\mathrm{d}#4}

% Inner product
\newcommand{\inner}[2]{\left\langle#1,#2\right\rangle}

% Absolute value
\newcommand{\abs}[1]{\left \vert #1 \right \vert }

% If else (ternary operator)
\newcommand{\ifelse}[3]{#1~\textrm{if}~#2~\textrm{else}~#3}

% Vector-Arrow
\newcommand{\vecarrow}[1]{\vec{#1}}

% Vector 2 dimensions
\newcommand{\vtwo}[2]{\left(\begin{array}{c}#1\\#2\end{array}\right)}

% Vector 3 dimensions
\newcommand{\vthree}[3]{\left(\begin{array}{c}#1\\#2\\#3\end{array}\right)}

% Large Limit bar for defined integral borders
\newcommand{\eval}[2]{\Biggr|_{#1}^{#2}}

% Divides symbol
\newcommand{\divides}{\vert}

% Changing greek letters
\let\uglyepsilon\epsilon % Still saving the previous one
\let\epsilon\varepsilon

\let\otherphi\phi % Still saving the previous one
\let\phi\varphi

% Images from inkscape

\newcommand{\incfig}[2][\columnwidth]{%
	\def\svgwidth{#1}
	\import{../figures/}{#2.pdf_tex}
}

\renewcommand*{\aa}{\mathbf{a}}

\renewcommand*{\Re}{\mathrm{Re}}
\renewcommand*{\Im}{\mathrm{Im}}

%Shortcuts
\renewcommand{\d}{\mathrm{d}}

% Mathematical Operators:
\DeclareMathOperator*{\supp}{supp} % support of a vector
\DeclareMathOperator*{\argmin}{argmin}
\DeclareMathOperator*{\spn}{span}
\DeclareMathOperator{\id}{id}
\DeclarePairedDelimiter\ceil{\lceil}{\rceil}
\DeclarePairedDelimiter\floor{\lfloor}{\rfloor}

% For tabulars
\newcommand{\first}[1]{\textbf{#1}}

% Quotation marks
\newcommand*{\quotes}[1]{``#1''}


% Comments
\newcommand*{\jakob}[1]{\color{red} \textbf{\emph{#1}} \color{black}}
\newcommand*{\elias}[1]{\color{blue} \textbf{\emph{#1}} \color{black}}
\newcommand*{\mario}[1]{\color{green} \textbf{\emph{#1}} \color{black}}

% Nice to have
\DeclareMathOperator{\err}{err}
\DeclareMathOperator{\op}{op}


\newcommand*{\StandardCaptionDimTable}[2]{Visualization of the results for 
fixed dim (dim = $#1$) and various scale tested for $n=#2$ realizations.}